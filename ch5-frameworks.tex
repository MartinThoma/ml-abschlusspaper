%!TEX root = pixel-wise-street-segmentation.tex

\section{Frameworks}\label{sec:frameworks}
nolearn was used in combination with Lasagne~\cite{sander_dieleman_2015_27878}
to train the models. Lasagne is based on Theano~\cite{Bergstra2010}. We also
used Caffe~\cite{Jia2014} for \glspl{CNN}, but skipped that approach as the
framework crashed very often for different training approaches without giving
meaningful error messages.

Theano is a Python package which allows symbolic computation of derivatives as
well as automatic generation of GPU code. This is used to calculate the weight
update function for arbitrary feed-forward networks. Lasagne makes using Theano
simpler by providing some basic layer types like fully connected layers,
convolutional layers and pooling layers with their update function. nolearn
adds some syntactic sugar and neural network objects with a similar interface
as the \verb+scikit-learn+ package uses for its classifiers~\cite{scikit-learn}.