%!TEX root = pixel-wise-street-segmentation.tex

\subsection{SST}\label{sec:sst}
\Gls{sst} is a Python package hosted on \gls{PyPI} and developed on GitHub.
It makes use of Lasagne (see \cref{sec:frameworks}). It was mainly developed
during the course \enquote{Machine Learning Laboratory --- Applications} at
KIT by Sebastian Bittel, Vitali Kaiser, Marvin Teichmann and Martin Thoma.

\subsubsection{Installation}
\verb+sst+ can be installed via \verb+pip install sst+. To make the
installation as simple as possible, this does not try to install all
requirements.
\verb+sst selfcheck+ gives the user the possibility to check which packages
are still required and manually install them.

\subsubsection{Functionality}
\verb+sst+ makes use of Python files for neural network definitions. Those
models must have a \verb+generate_nnet(feature_vectors)+ function which
returns an object with a \verb+fit(features, labels)+ method, a
\verb+predict(feature_vector)+ method and a
\verb+predict_proba(feature_vector)+ method. This is typically achieved by
returning a \verb+nolearn+ model object. The Python network definition file
has to have two global variables: \verb+patch_size+ (a positive integer) and
\verb+fully+ (a Boolean). The first variable is the patch size expected by the
neural network, the second variable indicates if the neural network was trained
to classify each pixel of the complete patch (\verb+fully = True+) or only the
center pixel (\verb+fully = False+).

\verb+sst --help+ shows all subcommands. The subcommands are currently

\begin{itemize}
    \item \verb+selfcheck+: Test which components or Python packages are
                            missing and have to be installed to be able to use
                            all features of \verb+sst+.
    \item \verb+train+: Train a neural network.
    \item \verb+eval+: Evaluate a trained network on a photograph and also
                       generate an overlay image of the segmentation and the
                       data photograph.
    \item \verb+serve+: Start a web server which lets the user choose images
                        from the local file system to predict the label and to
                        show overlays.
    \item \verb+view+: Show all information about an existing model.
    \item \verb+video+: Generate a video.
\end{itemize}