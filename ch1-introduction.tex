%!TEX root = pixel-wise-street-segmentation.tex

\section{Introduction}
Pixel-wise segmentation of street is an important part of assisted and
autonomous driving~\cite{Tarel2009}. It can help to understand road scenes and
reduce the space to search for lane markings. Traditionally, road segmentation
is done with computer vision methods such as watershed
transformation~\cite{Beucher1990}. Recent advances in deep neural networks,
especially in computer vision, suggest that \glspl{CNN} might be able to
achieve higher classification accuracy on road segmentation tasks then those
traditional approaches.

The algorithms need to be fast (e.g. classify an image in less than
\SI{0.3}{\second}) to be applicable in self-driving cars.

\Cref{sec:related-work} mentions published work which influenced us in the
choice of our methods. The basics methods used are explained in
TODO-LINK-CONCEPT-VITALI. It follows a description of the realization in
\cref{sec:realization} with a description of the used frameworks as well as
details about the developed SST framework. Models are explained in
\cref{sec:model} and evaluated in \cref{sec:evaluation}. Finally,
\cref{sec:discussion} summarizes the lessons we've learned and mentiones how
our work can be continued.