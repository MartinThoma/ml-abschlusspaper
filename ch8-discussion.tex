%!TEX root = write-math-ba-paper.tex

\section{Discussion}\label{sec:discussion}

In standard scenes, the classification accuracy is impressive. The street gets
segmented very well in a runtime of well below  $\SI{0.5}{\second}$. However,
in some images the model does perform very badly. These are mainly images where
with special situations such as an uncommon street color and unusual
lightning. We believe that this problems can be easily eliminated by using more
training data. One approach to get better results on the kitty data set is to
train the model with different data and only use the kitty training data for
fine-tuning.

One advantage of our models is that they are perfectly parallelizable. Each
image section can be evaluated independently. This can be adventurous in
practical applications. When using specialized hardware such as neuromorphic
chips it is possible to build hundreds of cores in a car. In such a case our
classification approach can yield outstanding results. Given enough training
data (e.g. 1.2~million images) using GoogLeNet or AlexNet can provide perfect
classification results.

Finally one can also improve the results with better hardware. For some of our
models the \gls{GPU} RAM was the limiting factor. Especially for the regression
model using bigger image section can lead to much better results in quality as
well as in speed.

