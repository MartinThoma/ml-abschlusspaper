%!TEX root = pixel-wise-street-segmentation.tex

\section{Related Work}\label{sec:related-work}
Road segmentation is a sub problem of general scene parsing or segmentation. In scene parsing every object in a scene is classified pixelwise with a label. Whereas in road segmentation often only two classes exist and more assumptions can be applied.\\
In the early days roads where often annotated by color-based histogram approaches and specific model knowledge. Examples are the in 1994 introduced approach \cite{Beucher1990} using the watershed algorithm or \cite{aly2008real} where roads were annoted indirectly by lane markings found with a hough transformation.\\
Later insights of general scene parsing where transferred and more generic approaches like [todo] have achieved remarkable results with a Markov Random Field (MRF) and superpixels.\\
The impressive classification results of Convolutional Neural Networks (CNNs) like AlexNet \cite{krizhevsky2012imagenet} or GoogleNet \cite{SzegedyLJSRAEVR14} during the the Google ImageNet LSVRC-2010 contest, made CNNs interesting for all kinds of computer vision problems like e.g. segmentation. \\
With \cite{long2014fully} Long and his team introduced a method for general scene parsing based on Fully Convolutional Neural Networks (FCNNs) and deconvolutional layer.\\
This approach is used as a blueprint of our implementation, described in Section \ref{sec:model}. Therefore the main concepts are introduced at the end of this section.\\
In \cite{mohan2014deep} a approach is presented, which makes also use of a FCNN in combination with deconvolution. In comparison the Long´s network, it is less deep and uses less convolutional then deconvolutional layers. Furthermore they divide the input in multiple patches and train separate neural network for each. Their model achieved the best-recorded result on the same dataset we use, which is described in Section \ref{sec:datasets}.    

\subsection{CNN}
A CNN is bla.

\subsection{FCNN}
Is CNN where bla.

\subsection{Deconvolutional Layer}
Deconv are mainly inversed convultions. Can be seen as inversed conv.
